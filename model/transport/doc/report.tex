\documentclass{warpdoc}
\newlength\lengthfigure                  % declare a figure width unit
\setlength\lengthfigure{0.158\textwidth} % make the figure width unit scale with the textwidth
\usepackage{psfrag}         % use it to substitute a string in a eps figure
\usepackage{subfigure}
\usepackage{rotating}
\usepackage{pstricks}
\usepackage[innercaption]{sidecap} % the cute space-saving side captions
\usepackage{scalefnt}
\usepackage{amsbsy}
\usepackage{bm}
\usepackage{amsmath}

%%%%%%%%%%%%%=--NEW COMMANDS BEGINS--=%%%%%%%%%%%%%%%%%%%%%%%%%%%%%%%%%%
\newcommand{\alb}{\vspace{0.1cm}\\} % array line break
\newcommand{\rhos}{\rho}
\newcommand{\cv}{{c_v}}
\newcommand{\cp}{{c_p}}
\newcommand{\Sct}{{{\rm Sc}_{\rm T}}}
\newcommand{\Prt}{{{\rm Pr}_{\rm T}}}
\newcommand{\nd}{{{n}_{\rm d}}}
\newcommand{\ns}{{{n}_{\rm s}}}
\newcommand{\nn}{{{n}_{\rm n}}}
\newcommand{\ndm}{{\bar{n}_{\rm d}}}
\newcommand{\nsm}{{\bar{n}_{\rm s}}}
\newcommand{\turb}{_{\rm T}}
\newcommand{\mut}{{\mu\turb}}
\newcommand{\mfa}{\scriptscriptstyle}
\newcommand{\mfb}{\scriptstyle}
\newcommand{\mfc}{\textstyle}
\newcommand{\mfd}{\displaystyle}
\newcommand{\hlinex}{\vspace{-0.34cm}~~\\ \hline \vspace{-0.31cm}~~\\}
\newcommand{\hlinextop}{\vspace{-0.46cm}~~\\ \hline \hline \vspace{-0.32cm}~~\\}
\newcommand{\hlinexbot}{\vspace{-0.37cm}~~\\ \hline \hline \vspace{-0.50cm}~~\\}
\newcommand{\tablespacing}{\vspace{-0.4cm}}
\newcommand{\fontxfig}{\footnotesize\scalefont{0.918}}
\newcommand{\fontgnu}{\footnotesize\scalefont{0.896}}
\renewcommand{\fontsizetable}{\footnotesize\scalefont{1.0}}
\renewcommand{\fontsizefigure}{\footnotesize}
%\renewcommand{\vec}[1]{\pmb{#1}}
%\renewcommand{\vec}[1]{\boldsymbol{#1}}
\renewcommand{\vec}[1]{\bm{#1}}
\setcounter{tocdepth}{3}
\let\citen\cite
\newcommand\frameeqn[1]{\fbox{$\displaystyle #1$}}

%%%%%%%%%%%%%=--NEW COMMANDS BEGINS--=%%%%%%%%%%%%%%%%%%%%%%%%%%%%%%%%%%

\setcounter{tocdepth}{3}

%%%%%%%%%%%%%=--NEW COMMANDS ENDS--=%%%%%%%%%%%%%%%%%%%%%%%%%%%%%%%%%%%%



\author{
  Bernard Parent
}

\email{
  bernparent@gmail.com
}

\department{
  Aerospace and Mechanical Engineering	
}

\institution{
  University of Arizona
}

\title{
  Transport Coefficients
}

\date{
  1998--2015
}

%\setlength\nomenclaturelabelwidth{0.13\hsize}  % optional, default is 0.03\hsize
%\setlength\nomenclaturecolumnsep{0.09\hsize}  % optional, default is 0.06\hsize

\nomenclature{

  \begin{nomenclaturelist}{Roman symbols}
   \item[$a$] speed of sound
  \end{nomenclaturelist}
}


\abstract{
abstract
}

\begin{document}
  \pagestyle{headings}
  \pagenumbering{arabic}
  \setcounter{page}{1}
%%  \maketitle
  \makewarpdoctitle
%  \makeabstract
  \tableofcontents
%  \makenomenclature
%%  \listoftables
%%  \listoffigures







\section{Dixon-Lewis Transport Coefficients}

The molecular
viscosity of a mixture of gases $\eta$ is computed using Wilke's mixing rule:
%
\begin{equation}
\eta= \mfd\sum_{\mfa k=1}^{\ns}   \frac{\mfd \eta_k {\chi}_k}
                     {\mfd {\chi}_k +
                          \mfd\sum_{\mfa l=1}^{\ns}{\chi}_l \phi_{k,l}}
            ~~~~~~~~l \neq k
\label{eqn:molvisc-mixture}
\end{equation}
%
%
\begin{equation}
{\rm with~~~~~}\phi_{k,l}=\mfd \frac{ \left[  1 + \sqrt{ \mfd {\eta_k }/{\eta_l }}
                              \left( \mfd {{\cal M}_l}/{{\cal M}_k}  \right)^ \frac{1}{4}  \right]^2}
{ \sqrt{\mfd 8\left(1+\mfd {{\cal M}_k}/{{\cal M}_l}\right)}}
~~~~~~~~~~~
{\chi}_k= \mfd\frac{w_k / {\cal M}_k}
              {\mfd\sum_{\mfa l=1}^{\ns} \left( w_l / {\cal M}_l \right)}
\label{eqn:molvisc-phi}
\end{equation}
%
where $w_k$ is the mass fraction and $\chi_k$ the molar fraction. The molecular thermal conductivity of a mixture of gases
$\kappa$ is found similarly to the molecular dynamic
viscosity from the Mason and Saxena \cite{gen:mason-saxena} relation:
%
\begin{equation}
\kappa= \mfd\sum_{\mfa k=1}^{\ns}  \mfd \frac{\mfd\kappa_k \chi_k}
                     {\mfd \chi_k + \mfd 1.0654
                          \mfd\sum_{\mfa l=1}^{\ns}  {\chi}_l \phi_{k,l}}
            ~~~~~~~~l \neq k
\label{eqn:moltherm-mixture}
\end{equation}
%
%
\begin{table}[t]
\fontsizetable
\begin{center}
  \begin{threeparttable}
    \tablecaption{$\epsilon$ and $\sigma$ for some species \tnote{1}}
    \fontsizetable
    \begin{tabular}{lllll}
      \toprule
species$_k$     &  ${\rm He}$    &  ${\rm H}_2$   &  ${\rm O}_2$   &  ${\rm N}_2$  \\
\midrule
$\epsilon_k$ [K]&  $10.22$       &  $59.7$        &  $106.7$       &  $71.4$       \\
$\sigma_k$ [nm] &  $0.2576$      &  $0.2827$      &  $0.3467$      &  $0.3798$     \\
      \bottomrule
    \end{tabular}
    \label{table:species-epssig}
    \begin{tablenotes}
      \item [1] taken from Dixon-Lewis; $T^\star_k=T/ \epsilon_k$
    \end{tablenotes}
  \end{threeparttable}
\end{center}
\end{table}
%
The molecular diffusion coefficient of species $k$, $\nu_k$,
is found from the kinetic theory of gases as outlined in Dixon-Lewis \cite{gen:dixon-lewis}:
%
\begin{equation}
\mfd\nu_k= \frac{\rhos \left( 1 - {\chi}_k \right)}
            {\mfd\sum_{\mfa l=1}^{\ns} \frac{{\chi}_l}{{\cal D}_{k,l}} +1.0E-20   }
            ~~~~~~~{\rm for}~~~l \neq k
\label{eqn:moldiff-nu}
\end{equation}
%
where ${\cal D}_{k,l}$ is the binary diffusion coefficient, or
a measure of how much gas $k$ diffuses into gas $l$.
We now need the polynomials taken from the literature necessary to determine
${\cal D}_{k,l}$, $\eta_k$ and $\kappa_k$.
The species dynamic viscosity $\eta_k$ is
derived from kinetic theory assuming that species $k$ is a pure gas:
%
\begin{equation}
\eta_k= \left( 8.44107 \times 10^{-7} \right) \frac{  \sqrt{{\cal M}_k T}}
                          {\sigma_k^2 \Omega^{22}_k}
\label{eqn:molvisc-muk1}
\end{equation}
%
%
\begin{table*}[t]
\fontsizetable
\begin{center}
  \begin{threeparttable}
    \tablecaption{Polynomial constants needed to determine $\Omega^{22}$ 
             $\Omega^{11}$ and $A^\star$ taken from Dixon-Lewis}
    \fontsizetable
    \begin{tabular*}{\textwidth}{@{\extracolsep{\fill}}llllll}
      \toprule
$T^\star_k$& $d_0$          & $d_1$           & $d_2$          & $d_3$           & $d_4$ \\
\midrule
$<5$       & $2.3527333$E$+0$ & $-1.3589968$E$+0$ & $5.2202460$E$-1$ & $-9.4262883$E$-2$ & $6.4354629$E$-3$ \\
$5-10$     & $1.2660308$E$+0$ & $-1.6441443$E$-1$ & $2.2945928$E$-2$ & $-1.6324168$E$-3$ & $4.5833672$E$-5$ \\
$>10$      & $8.5263337$E$-1$ & $-1.3552911$E$-2$ & $2.6162080$E$-4$ & $-2.4647654$E$-6$ & $8.6538568$E$-9$ \\
\midrule
$T^\star_k$& $e_0$          & $e_1$           & $e_2$           &~&~\\
\midrule
$<5$       & $1.1077725$E$+0$ & $-9.4802344$E$-3$ & $+1.6918277$E$-3$ &~&~ \\
$5-10$     & $1.0871429$E$+0$ & $+3.1964282$E$-3$ & $-8.9285689$E$-5$ &~& ~\\
$>10$      & $1.1059000$E$+0$ & $+6.5136364$E$-4$ & $-3.4090910$E$-6$ &~&~ \\
      \bottomrule
    \end{tabular*}
    \label{table:species-Omega}
    \begin{tablenotes}
      \item  $\Omega^{22}(T^\star_k)= A^\star (T^\star_k) \Omega^{11}(T^\star_k)$ 
      \item  $\Omega^{11}(T^\star_k)=d_0+d_1 {T^\star_k}+d_2 {T^\star_k}^2+d_3 {T^\star_k}^3+d_4 {T^\star_k}^4$ 
      \item  $A^\star (T^\star_k)=e_0+e_1 {T^\star_k}+e_2 {T^\star_k}^2$ 
    \end{tablenotes}
  \end{threeparttable}
\end{center}
\end{table*}
%
where $\sigma_k$ is the collision diameter, tabulated for all species,
$\Omega^{22}_k$ is the reduced
collision integral involving polynomials function only of
$T^\star_k$
which is the reduced temperature corresponding to
$T/ \epsilon_k$ with $\epsilon_k$ being the maximum energy of
attraction between colliding molecules for species $k$ (tabulated
versus T). The thermal conductivity of species $k$, $\kappa_k$ is determined
differently depending on whether
species $k$ is polyatomic (such as H$_2$O) or monoatomic
(such as H and O); for a monoatomic gas, only the
translational energy of the particle is considered,
neglecting the vibrational energy. For a polyatomic gas,
both the translational and vibrational energies are taken into
account, the Eucken correction being used. Hence $\kappa_k$
is found from:
%
\begin{equation}
\kappa_k = \left\{ \begin{array}{ll}
\vspace{0.2cm}
\mfd\frac{15}{4} \mfd\frac{\cal R}{{\cal M}_k} \eta_k & {\rm for~a~monatomic~gas} \\
\mfd\frac{15}{4} \mfd\frac{\cal R}{{\cal M}_k} \eta_k \left(  0.115 +
   \mfd\frac{ \left( 0.354 \right) {\cp}_k{\cal M}_k}{\cal R}  \right)
        & {\rm for~a~polyatomic~gas} \\
\end{array}
\right.
\label{eqn:moltherm-kappa}
\end{equation}
%
The binary diffusion coefficient of species $k$ with respect to species $l$,
${\cal D}_{k,l}$, is given by:
%
\begin{equation}
{\cal D}_{k,l}=\left( 2.381112 \times 10^{-5} \right) \frac{\sqrt{T^3 \mfd  \left(\frac{1}{{\cal M}_l} + \frac{1}{{\cal M}_k} \right)}}
              {  \left( \sigma_k + \sigma_l \right)^2 P \Omega^{11}_{k,l}}
\label{eqn:moldiff-binary}
\end{equation}
%
where $\Omega^{11}_{k,l}$ involves
polynomials dependent on the reduced temperature $T^\star_k$
and can be found in Dixon-Lewis, noting that in this case, $T^\star=T/\epsilon_{k,l}$
where $\epsilon_{k,l}=\sqrt{\epsilon_k \epsilon_l}$; the units are Pascal for $P$, nm for $\sigma$,
K for $T$ and $m^2/s$ for $\cal D$.




\section{Transport Coefficients of Fully-Ionized Plasmas}



\subsection{Ion Mobility in Fully-Ionized Plasma}

We can obtain the mobility of ions in a fully ionized plasma assuming the only collisions of importance are ion-ion collisions:
%
\begin{equation}
 \mu_{\rm i} = \frac{|C_i|}{m_i \nu_{\rm ii}}
\end{equation}
% 
where the ion-ion collision frequency corresponds to \cite{book:1984:chen}:
%
\begin{equation}
\nu_{\rm ii}=\xi \frac{ N_{\rm i}}{ \sqrt{m_{\rm i}} (k_{\rm B} T_{\rm i})^\frac{3}{2}}  \frac{C_{\rm i}^4}{16 \pi^2 \epsilon_0^2}  \ln \Lambda
\end{equation}
%
with $\xi$ a non-dimensional constant equal to 1.71 (Macheret), 2.36 (Ian Hutchinson). As well, $\ln \Lambda$ is another constant in the range 5-7.


After substituting the latter in the former:
%
\begin{equation}
\frameeqn{
 \mu_{\rm i} = \frac{16 \pi^2 \epsilon_0^2 (k_{\rm B} T_{\rm i})^\frac{3}{2}}{\mfd \sqrt{m_{\rm i}} \xi  N_{\rm i}  |C_{\rm i}|^3  \ln \Lambda}
}
\end{equation}
% 
Setting $\xi$ to 1.71 and $\ln \Lambda$ to 6.3, and assuming that the ion has a single charge, the latter can be rewritten as:
%
\begin{equation}
\frameeqn{
 \mu_{\rm i} = 14.3 m_{\rm i}^{-0.5} T_{\rm i}^{1.5} N_{\rm i}^{-1}
}
\end{equation}
% 
where $m_{\rm i}$ is the ion mass in kg, $T_{\rm i}$ the ion translational temperature in K, $N_{\rm i}$ the total ion number density in $\rm m^{-3}$, and $\mu_{\rm i}$ the ion mobility in m$^2$/Vs.


\subsection{Electron Mobility in Fully-Ionized Plasma}

We can obtain the mobility of electrons in a fully ionized plasma assuming the only collisions of importance are electron-ion collisions:
%
\begin{equation}
 \mu_{\rm e} = \frac{|C_e|}{m_e \nu_{\rm ei}}
\end{equation}
% 
where the electron-ion collision frequency corresponds to \cite{book:1984:chen}:
%
\begin{equation}
\nu_{\rm ei}=\xi \frac{ \sqrt{2} N_{\rm i}}{ \sqrt{m_{\rm e}} (k_{\rm B} T_{\rm e})^\frac{3}{2}}  \frac{C_{\rm e}^4}{16 \pi^2 \epsilon_0^2}  \ln \Lambda
\end{equation}
%
with $\xi$ a non-dimensional constant equal to 1.71 (Macheret), 2.36 (Ian Hutchinson). As well, $\ln \Lambda$ is another constant in the range 5-7.


After substituting the latter in the former:
%
\begin{equation}
\frameeqn{
 \mu_{\rm e} = \frac{16 \pi^2 \epsilon_0^2 (k_{\rm B} T_{\rm e})^\frac{3}{2}}{\mfd \sqrt{2}\sqrt{m_{\rm e}} \xi  N_{\rm i}  |C_{\rm e}|^3  \ln \Lambda}
}
\end{equation}
% 

Setting $\xi$ to 1.71 and $\ln \Lambda$ to 6.3, the latter can be rewritten as:
%
\begin{equation}
\frameeqn{
 \mu_{\rm e} = 10.1 m_{\rm e}^{-0.5} T_{\rm e}^{1.5} N_{\rm i}^{-1}
}
\end{equation}
% 
where $m_{\rm e}$ is the electron mass in kg, $T_{\rm e}$ the electron  temperature in K, $N_{\rm i}$ the total ion number density in $\rm m^{-3}$, and $\mu_{\rm e}$ the ion mobility in m$^2$/Vs.



\subsection{Ion Viscosity in a Fully-Ionized Plasma}

From the 2002 Revised NRL Plasma Formulary \cite{nrl:2002:huba}, the ion viscosity in a fully ionized plasma corresponds to:
%
\begin{equation}
\eta_{\rm i}= 0.96    \frac{N_{\rm i} k_{\rm B} T_{\rm i} }{\nu_{\rm ii}}
\end{equation}
%
But the ion mobility corresponds to:
%
\begin{equation}
 \mu_{\rm i} = \frac{|C_i|}{m_i \nu_{\rm ii}}
\end{equation}
%
Isolate $\nu_{\rm ii}$ and substitute the latter in the viscosity:
%
\begin{equation}
\eta_{\rm i}=  0.96  \frac{N_{\rm i} k_{\rm B} T_{\rm i} m_{\rm i} \mu_{\rm i}}{|C_i|}
\end{equation}
%
Substitute the mobility, assume that the ion is singly charged ($C_i=e$) and simplify:
%
\begin{equation}
\frameeqn{
\eta_{\rm i}=   0.00118  \sqrt{m_{\rm i}}   T_{\rm i}^{2.5}  
}
\end{equation}
%
In the latter, $m_{\rm i}$ is the ion mass in kg, $T_{\rm i}$ is the ion translational temperature in K, and $\eta_{\rm i}$ the ion viscosity in kg/ms. 



\subsection{Electron Viscosity in a Fully-Ionized Plasma}

From the 2002 Revised NRL Plasma Formulary \cite{nrl:2002:huba}, the electron viscosity in a fully ionized plasma corresponds to:
%
\begin{equation}
\eta_{\rm e}= 0.73    \frac{N_{\rm i} k_{\rm B} T_{\rm e} }{\nu_{\rm ei}}
\end{equation}
%
In deriving the latter, only the electron-ion collisions are taken into consideration because the electron-electron collisions would not result in significant momentum exchange.

Now, note that the electron mobility corresponds to:
%
\begin{equation}
 \mu_{\rm e} = \frac{|C_e|}{m_e \nu_{\rm ei}}
\end{equation}
%
Isolate $\nu_{\rm ei}$ and substitute the latter in the viscosity:
%
\begin{equation}
\eta_{\rm e}= 0.73    \frac{N_{\rm i} k_{\rm B} T_{\rm e}\mu_{\rm e} m_e }{|C_e|}
\end{equation}
%
Substitute the electron mobility:
%
\begin{equation}
\eta_{\rm e}= 7.37    \frac{ k_{\rm B} T_{\rm e}^{2.5}  \sqrt{m_e} }{|C_e|}
\end{equation}
%
Substitute constants:
%
\begin{equation}
\frameeqn{
\eta_{\rm e}=   6.35 \cdot 10^{-4}  \sqrt{m_{\rm e}}   T_{\rm e}^{2.5}  
}
\end{equation}
%
In the latter, $m_{\rm e}$ is the electron mass in kg, $T_{\rm i}$ is the ion translational temperature in K, and $\eta_{\rm i}$ the ion viscosity in kg/ms. 


\subsection{Ion Thermal Conductivity}

When deriving the ion energy transport equation from the first law of thermo and substituting the ion momentum equation (drift-diffusion), and reformatting, it can be shown that the ion thermal conductivity corresponds to (for an atomic ion):
%
\begin{equation}
\frameeqn{
 \kappa_{\rm i}=\frac{5 N_{\rm i} k_{\rm B}^2 T_{\rm i} \mu_{\rm i}}{2|C_{\rm i}|}
}
\end{equation}
%
We can now find the ion Prandtl number:
%
\begin{equation}
 {\rm Pr_i} = \frac{(c_p)_{\rm i} \eta_{\rm i}}{\kappa_{\rm i}}
\end{equation}
%
Substitute the ion viscosity and the ion thermal conductivity in the latter:
%
\begin{equation}
 {\rm Pr_i} = 0.96 (c_p)_{\rm i}  \frac{N_{\rm i} k_{\rm B} T_{\rm i} m_{\rm i} \mu_{\rm i}}{|C_i|} 
 \frac{2|C_{\rm i}|}{5 N_{\rm i} k_{\rm B}^2 T_{\rm i} \mu_{\rm i}}
\end{equation}
%
Simplify:
%
\begin{equation}
 {\rm Pr_i} = (c_p)_{\rm i}  
 \frac{2  m_{\rm i}}{15  k_{\rm B} }
\end{equation}
%
But for an atomic species without electronic energy the specific heat at constant pressure is equal to $\frac{5}{2}k_{\rm B}/m$:
%
\begin{equation}
 {\rm Pr_i} = 0.96 \frac{5 k_{\rm B}}{2 m_{\rm i}}
 \frac{2  m_{\rm i}}{5  k_{\rm B} }
\end{equation}
%
or
%
\begin{equation}
 {\rm Pr_i} = 0.96
\end{equation}
%



\subsection{Electron Thermal Conductivity}

When deriving the electron energy transport equation from the first law of thermo and substituting the electron momentum equation (drift-diffusion), and reformatting, it can be shown that the electron thermal conductivity corresponds to (in the absence of magnetic field):
%
\begin{equation}
\frameeqn{
 \kappa_{\rm e}=\frac{5 N_{\rm e} k_{\rm B}^2 T_{\rm e} \mu_{\rm e}}{2|C_{\rm e}|}
}
\end{equation}
%
We can now find the electron Prandtl number:
%
\begin{equation}
 {\rm Pr_e} = \frac{(c_p)_{\rm e} \eta_{\rm e}}{\kappa_{\rm e}}
\end{equation}
%
Substitute the electron viscosity and the electron thermal conductivity in the latter:
%
\begin{equation}
 {\rm Pr_e} = (c_p)_{\rm e}  6.35 \cdot 10^{-4}  \sqrt{m_{\rm e}}   T_{\rm e}^{2.5} \frac{2|C_{\rm e}|}{5 N_{\rm e} k_{\rm B}^2 T_{\rm e} \mu_{\rm e}}
\end{equation}
%
Substitute the electron mobility and simplify noting that $N_{\rm i} = N_{\rm e}$ for a quasi-neutral plasma:
%
\begin{equation}
 {\rm Pr_e} = 6.35 \cdot 10^{-4} (c_p)_{\rm e}    \sqrt{m_{\rm e}}    \frac{2|C_{\rm e}|}{5\cdot 10.1\cdot  k_{\rm B}^2   m_{\rm e}^{-0.5}  }
\end{equation}
%
But the specific heat at constant pressure is equal to $\frac{5}{2}k_{\rm B}/m_e$:
%
\begin{equation}
 {\rm Pr_e} = 6.35 \cdot 10^{-4} \cdot \frac{5}{2}  \cdot      \frac{2|C_{\rm e}|}{5\cdot 10.1\cdot  k_{\rm B}    }
\end{equation}
%
After substituting values for the constants and simplifying we get the following Prandtl number for the electrons:
%
\begin{equation}
\frameeqn{
 {\rm Pr_e} = 0.73
}
\end{equation}
%


\section{Transport Coefficients in a Partially-Ionized Plasma}

%
\begin{table*}[b]
  \center
  \begin{threeparttable}
    \tablecaption{Ion and electron mobilities in dry air.\tnote{a,d}}
    \label{tab:mobilities}
    \fontsizetable
    \begin{tabular*}{\textwidth}{l@{\extracolsep{\fill}}ll}
    \toprule
    Species & Mobility, $\rm m^2\cdot V^{-1}\cdot s^{-1}$  & Reference\\
    \midrule
    Air$^+$         & ${\rm avgh} \left(N_{\rm n}^{-1} \cdot {\rm min}\left(0.84\cdot 10^{23}\cdot T^{-0.5},~~2.35 \cdot 10^{12}\cdot \left(E^\star\right)^{-0.5}\right)~,~~14.3 \cdot m_{\rm air}^{-0.5} \cdot T^{1.5} \cdot N_{\rm i}^{-1} \right)$  & \cite{misc:1968:sinnott}\tnote{b}\alb
    N$_2^+$         & ${\rm avgh} \left(N_{\rm n}^{-1} \cdot {\rm min}\left(0.75\cdot 10^{23}\cdot T^{-0.5},~~2.03 \cdot 10^{12}\cdot \left(E^\star\right)^{-0.5}\right)~,~~14.3 \cdot m_{\rm N_2^+}^{-0.5} \cdot T^{1.5} \cdot N_{\rm i}^{-1} \right)$  & \cite{misc:1968:sinnott}\alb
    O$_2^+$         & ${\rm avgh} \left(N_{\rm n}^{-1} \cdot {\rm min}\left(1.18\cdot 10^{23}\cdot T^{-0.5},~~3.61 \cdot 10^{12}\cdot\left(E^\star\right)^{-0.5}\right)~,~~14.3 \cdot m_{\rm O_2^+}^{-0.5} \cdot T^{1.5} \cdot N_{\rm i}^{-1} \right)$  & \cite{misc:1968:sinnott}\alb
    O$_2^-$         & ${\rm avgh} \left(N_{\rm n}^{-1} \cdot {\rm min}\left(0.97 \cdot 10^{23}\cdot T^{-0.5},~~3.56 \cdot 10^{19} \cdot \left(E^\star\right)^{-0.1}\right)~,~~14.3 \cdot m_{\rm O_2^-}^{-0.5} \cdot T^{1.5} \cdot N_{\rm i}^{-1} \right)$  & \cite{misc:1983:gosho}\alb
    e$^-$         & ${\rm avgh}\left( N_{\rm n}^{-1} \cdot 3.74\cdot 10^{19} \cdot {\rm exp}\left(33.5 \cdot \left({\rm ln}\, T_{\rm e} \right)^{-0.5}\right) ~,~~N_{\rm i}^{-1} \cdot 1.061 \cdot 10^{16} \cdot T_{\rm e}^{1.5}  \right)$  & \cite[Ch.\ 21]{book:1997:grigoriev}\tnote{c}\\
    other~ions         & ${\rm avgh} \left(N_{\rm n}^{-1} \cdot {\rm min}\left(2.2 \cdot 10^{10}\cdot (m_{\rm i} T)^{-0.5},~~0.55 \cdot \left(m_{\rm i} E^\star\right)^{-0.5}\right)~,~~14.3 \cdot m_{\rm i}^{-0.5} \cdot T^{1.5} \cdot N_{\rm i}^{-1} \right)$  & --\alb
    \bottomrule
    \end{tabular*}
    \begin{tablenotes}
      \item[a] Notation and units:  $T_{\rm e}$ is in Kelvin; $T$ is in Kelvin; $N$ is the total number density of the plasma in 1/m$^3$; $N_{\rm i}$ is the sum of the positive ion number densities in 1/m$^3$;  $N_{\rm n}$ is the sum of the neutral species number densities  in 1/m$^3$;  $E^\star$ is the reduced effective electric field  ($E^\star \equiv |\vec{E}|/N$) in units of V$\cdot$m$^2$.
      \item[b] The ``air ion'' mobility is obtained from the N$_2^+$ and O$_2^+$ ion mobilities assuming a N$_2^+$:O$_2^+$ ratio of 4:1. 
      \item[c] The expression for the electron-neutral collisions approximates the data given in Chapter 21 of Ref.\ \cite{book:1997:grigoriev}; The equation can be used in the range $1000~{\rm K} \le T_{\rm e} \le 57900 ~{\rm K}$ with a relative error on the mobility not exceeding 20\%. In the range $287~{\rm K} \le T_{\rm e} < 1000~{\rm K}$, the relative error is less than 30\%. 
      \item[d] The function avgh($a$,$b$) returns the harmonic average of the two arguments as follows $1/(a^{-1}+b^{-1})$.
    \end{tablenotes}
   \end{threeparttable}
\end{table*}
%

\subsection{Electron Mobility}

Note that when the ionization fraction exceeds 0.001 or so, the electron mobility needs to be corrected to account for electron-ion collisions as follows:
%
\begin{equation}
\mu_{\rm e}=\frac{1}{\frac{1}{\mu_{\rm en}}+\frac{1}{\mu_{\rm ei}}}
\end{equation}
%
with the electron mobility with respect to ions is set to (see \cite[page 180--181]{book:1984:chen}):
%
\begin{equation}
\mu_{\rm ei} = \frac{9}{4\pi\sqrt{3}} \frac{(4\pi\epsilon_0)^2 (k_{\rm B}T_{\rm e})^{1.5}}{m_{\rm e}^{0.5} e^3 N_{\rm i} \ln \Lambda}
\end{equation}
%
where $e$ is the elementary charge, $N_{\rm i}$ the positive ion number density (sum of all positive ions), $m_{\rm e}$ the electron mass, and $\ln \Lambda$ varies between 5 and 8 depending on the type of ion. Setting $\ln \Lambda$ to 6.3, we obtain:
%
\begin{equation}
\mu_{\rm ei}= \frac{1.061 \times 10^{16}}{\rm m\cdot K^{1.5}\cdot V \cdot s} \frac{T_{\rm e}^{1.5}}{N_{\rm i}}
\end{equation}
%
Also, we can fit a polynomial through the experimental data outlined in in Chapter 21 of Ref.\ \cite{book:1997:grigoriev} to find $\mu_{\rm en}$ in units of $\rm m^2\cdot V^{-1}\cdot s^{-1}$:
%
\begin{equation}
\mu_{\rm en}=N_{\rm n}^{-1} \cdot 3.74\cdot 10^{19} \cdot {\rm exp}\left(33.5 \cdot \left({\rm ln}\, T_{\rm e} \right)^{-0.5}\right)
\end{equation}
%
where $N_{\rm n}$ is the number density of the neutrals in $m^{-3}$ and $T_{\rm e}$ is the electron temperature in K.
The latter equation can be used in the range $1000~{\rm K} \le T_{\rm e} \le 57900 ~{\rm K}$ with a relative error on the mobility not exceeding 20\%. In the range $287~{\rm K} \le T_{\rm e} < 1000~{\rm K}$, the relative error is less than 30\%. 


\subsection{Ion Mobility}

Note that when the ionization fraction exceeds 0.001 or so, the ion mobility needs to be corrected to account for ion-ion collisions as follows:
%
\begin{equation}
\mu_{\rm i}=\frac{1}{\frac{1}{\mu_{\rm in}}+\frac{1}{\mu_{\rm ii}}}
\end{equation}
%
where $\mu_{\rm ii}$ is the ion mobility in the limit of a fully ionized plasma and $\mu_{\rm in}$ is the ion mobility in the limit of a weakly-ionized plasma. The ion mobility for a fully ionized plasma is found from the collision frequency of ion-ion collisions only and corresponds to:
%
\begin{equation}
 \mu_{\rm ii} = \frac{16 \pi^2 \epsilon_0^2 (k_{\rm B} T_{\rm i})^\frac{3}{2}}{\mfd \sqrt{m_{\rm i}} \xi  N_{\rm i}  |C_{\rm i}|^3  \ln \Lambda}
\end{equation}
% 
where $\xi$ is a constant that we here set to 1.71 and $\ln \Lambda$ is another constant that we here set to 6.3. For a single charged ion, the latter can be rewritten as:
%
\begin{equation}
 \mu_{\rm ii} = 14.3 m_{\rm i}^{-0.5} T_{\rm i}^{1.5} N_{\rm i}^{-1}
\end{equation}
% 
where $m_{\rm i}$ is the ion mass in kg, $T_{\rm i}$ the ion translational temperature in K, $N_{\rm i}$ the total ion number density in $\rm m^{-3}$, and $\mu_{\rm i}$ the ion mobility in m$^2$/Vs.


We can find the ion mobility for a weakly-ionized plasma $\mu_{\rm in}$ through the collision frequency between ions and neutrals. For instance, for the nitrogen ion $\rm N_2^+$, the mobility in air can be expressed as:
%
\begin{equation}
\mu_{\rm in}=N_{\rm n}^{-1} \cdot {\rm min}\left(2.2\cdot 10^{10} \cdot (m_{\rm i} T)^{-0.5},~~0.55  \cdot \left(m_{\rm i} E^\star\right)^{-0.5}\right)
\end{equation}
%
where $N_{\rm n}$ is the number density of the neutrals in $m^{-3}$ and $T$ is the ion translational temperature in K.

The mobility of a partially-ionized plasma can be obtained simply by taking the harmonic mean of the mobility of a fully-ionized plasma and the mobility of a weakly-ionized plasma. 

The mobilities of the various charged species are tabulated in Table \ref{tab:mobilities}. 



\subsection{Thermal Conductivity and Viscosity}

Whether the plasma is fully-ionized or weakly-ionized, the thermal conductivity can be expressed as a function of the mobility as follows:
%
\begin{equation}
\kappa_k =     \frac{ \rho_k k_{\rm B}  (c_p)_k T_k \mu_k}{|C_k|} 
\end{equation}
%
where $(c_p)_k$ is the specific heat at constant pressure of the $k$th species.

The scalar viscosity can be obtained from the thermal conductivity as follows:
%
\begin{equation}
\eta_k = \frac{{\rm Pr} \cdot \kappa_k}{(c_p)_k}
\end{equation}
% 
where Pr is the Prandtl number which is set to 0.73 for the electrons and to 0.96 for the ions.








\appendix


  \bibliographystyle{warpdoc}
  \bibliography{all}


\end{document}







Old stuff

electron energy loss function: 
- not used anymore but kept just in case
- see derivation in prim/fluid/doc/Transport_Equations/report.pdf

%
\begin{table}
  \center\fontsizetable
  \begin{threeparttable}
    \tablecaption{Polynomial coefficients needed for the electron energy loss function $\zeta_{\rm e}=k_0+k_1 T_{\rm e}+k_2 T_{\rm e}^2 + k_3 T_{\rm e}^3 + k_4 T_{\rm e}^4 + k_5 T_{\rm e}^5+ k_6 T_{\rm e}^6$.\tnote{a,b,c}} 
    \label{tab:xicoefficients}
    \fontsizetable
    \begin{tabular}{llll}
    \toprule
      Coefficient & Value for $T_{\rm e}<19444$~K & Value for $T_{\rm e}\ge 19444$~K  \\
    \midrule
      $k_0$          & $+5.1572656\times 10^{-4}$ & $+2.1476152\times 10^{-1}$  \\
      $k_1$, 1/K     & $+3.4153708\times 10^{-8}$ & $-4.4507259\times 10^{-5}$  \\
      $k_2$, 1/K$^2$ & $-3.2100688\times 10^{-11}$ & $+3.5155106\times 10^{-9}$ \\
      $k_3$, 1/K$^3$ & $+1.0247332\times 10^{-14}$ & $-1.3270119\times 10^{-13}$ \\
      $k_4$, 1/K$^4$ & $-1.2153348\times 10^{-18}$ & $+2.6544932\times 10^{-18}$ \\
      $k_5$, 1/K$^5$ & $+7.2206246\times 10^{-23}$ & $-2.7145800\times 10^{-23}$ \\
      $k_6$, 1/K$^6$ & $-1.4498434\times 10^{-27}$ & $+1.1197905\times 10^{-28}$ \\
    \bottomrule
    \end{tabular}
 \begin{tablenotes}
   \item[a] The expression for $\zeta_{\rm e}$ can be used in the range $0<T_{\rm e}<60000$~K
   \item[b] In the range $287~{\rm K}<T_{\rm e}<1500$~K, the relative error that the loss function induces on the electron temperature is not more than 30\%.
   \item[c] In the range $1500~{\rm K}<T_{\rm e}<57000$~K, the relative error that the loss function induces on the electron temperature is not more than 5\%.
 \end{tablenotes}
   \end{threeparttable}
\end{table}
%

